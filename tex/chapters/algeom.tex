
\chapter{ Algebraic analysis of focal lengths computation}

\label{seq:algeom}

In this section we analyze the methods for computing focal lengths from point correspondences by the algebraic geometry techniques. We show that indeed three and only three fundamental matrices (some possibly complex or of rank 1) can be derived using algebraic geometry. We analyze the degeneracies of the computation. We also show that three different formulae exist for computing focal lengths from a Fundamental matrix. One of these is the Bougnoux formula~\ref{eq:bougnoux1}, and two others weren't known before.
%We show that using the appropriate formula each time, a known degeneracy can be avoided. 

\section{Analysis}
We analyze the set of valid fundamental matrices of  camera pairs  calibrated up to focal lengths using techniques of algebraic geometry. To this end, we use Macaulay2~\cite{Macaulay}, a programming language and also a software pack for algebraic geometry and abstract commutative algebra. Macaulay2 tends to have an intuitive interface for many algebraic operations and, in general, code can be read as easily as mathematical equations. For this reason, instead of mathematical language, we will use in this section Macaulay2 code snippets directly. This helps to make our results reproducible, too.


In algebraic terms, we choose to describe the set of valid $\mathtt{F}$s as a variety in a field of rational numbers $\Q^{11}$ of 11 unknowns, 9 for elements of the fundamental matrix F and 2 for focal lengths $f_1$, $f_2$. We assume that the principal points $\mathbf{p}_1$, $\mathbf{p}_2$ are zero.  

The ideal $Gs$ is the ideal generated by rank \ref{eq:rankF} and trace constraints \ref{eq:trace} and saturated by the ideal $\langle f_1\,f_2 \rangle$. The saturation is desirable, because it can remove large spurious components corresponding to cases when $f_1=0$ or $f_2=0$, which can't happen in any real camera systems.

Below is a snippet of Macaulay2 code explaining the construction of  $Gs$.

\begin{verbatim}
R = QQ[f1,f2,f11,f12,f13,f21,f22,f23,f31,f32,f33, MonomialOrder=>Lex]
F = matrix{{f11,f12,f13},{f21,f22,f23},{f31,f32,f33}}
K1 = matrix{{f1, 0, 0}, {0, f1, 0}, {0, 0, 1}}
K2 = matrix{{f2, 0, 0}, {0, f2, 0}, {0, 0, 1}}
E = transpose(K2)*F*K1 -- Essential matrix
Et = transpose E
G = ideal(det(E)) + minors(1, 2*E*Et*E - trace(E*Et)*E); 
dim G, codim G, degree G
Gs = saturate(G,ideal(f1*f2)); -- det(K1),det(K2) are non-zero
dim Gs, codim Gs, degree Gs -- dimension, codimension and degree
\end{verbatim}

The ideal $Gs$ contains constraints under which a matrix is generically a fundamental matrix. The variety $\textbf{V}(Gs)$ therefore is a variety that contains all valid fundamental matrices. It also contains more matrices, specifically those of rank 0 and 1. Generically, however, we expect a matrix from the variety to be of rank 2. Moreover, when considered in the algebraically closed field $\C$, the variety does contain some complex matrices. In practical situations however, when considering a solver constructed this way, we can just sort out the spurious solutions afterwards as there will be only three solutions in total.

The 7pt algorithm can be regarded as intersecting the variety $\textbf{V}(Gs)$ it with 7 hyperplanes, and it gives us three different solutions. The solutions are actually one-dimensional subspaces, and therefore we would expect the ideal $Gs$ to have dimension 8, which it indeed has.
%\TP{NO. We would expect this to be 8 because the solutions are one-dimensional subspaces since all is homogeneous and we are in a projective space.} and degree 3, but in reality, as constructed, it has dimension of 8 and degree of 58. This is explained by the fact that the variety $\textbf{V}(Gs)$ has an additional component of dimension 8 and degree 58. 
%\OR{there is something wrong with this degree 58 component, but i'd better leave this out altogether}
%Executing 7pt algorithm can be viewed as adding epipolar constraints to the ideal, and later we will see that this component doesn't survive the procedure.

The next snippet shows how to compute algebraic conditions in terms of elements of Fundamental matrix only, by eliminating $f_1,f_2$.

\begin{verbatim}
M = eliminate(Gs,{f1,f2}); 
dim M, codim M, degree M
-- the commands "mingens gb" give a minimal set of generators 
-- for the groebner basis of an ideal 
m = mingens gb M
\end{verbatim}

After executing the above, we see that the ideal $M$ has only one generator - the $\det(\mathtt{F})$ polynomial. This means that the algebraic constraints on the set of fundamental matrices of up to focal lengths calibrated camera pairs are the same as for the fully uncalibrated camera case. It can be deduced that a seven-tuple of corresponding points obtained by completely uncalibrated cameras  can  also  be  explained  by  cameras  with  two  unknown focal lengths when the focal lengths are allowed to attain non-real values.

It also means that only the rank constraint~\ref{eq:rank} is needed to solve the system, and the trace constraint is extraneous. The trace constraint, however, allow us to compute the focal lengths.

\section{Computing focal lengths}
\label{sec:focalformulae}
In the next snippet we show how the Bougnoux formula~\cite{Bougnoux} can be derived with algebraic geometry, given as a polynomial in the entries of F. This is done by eliminating one of the focal lengths from the ideal $Gs$. It turns out that the Gr\"obner basis of the eliminated ideal $s_i$ contains the determinant of $\M{F}$ and additional three polynomials, from which a formula for computing the focal length that wasn't eliminated can be deducted. This means that there exist three algebraically independent constraints on each focal length.

\begin{verbatim}
s2 = mingens gb eliminate(Gs,f1)
s1 = mingens gb eliminate(Gs,f2)
-- Formulae for f1
(m11,c11) = coefficients(s1_1_0,Variables=>{f1}) -- extract coefficients 
(m12,c12) = coefficients(s1_2_0,Variables=>{f1}) -- extract coefficients
(m13,c13) = coefficients(s1_3_0,Variables=>{f1}) -- extract coefficients
-- Formulae for f2
(m21,c21) = coefficients(s2_1_0,Variables=>{f2}) -- extract coefficients 
(m22,c22) = coefficients(s2_2_0,Variables=>{f2}) -- extract coefficients
(m23,c23) = coefficients(s2_3_0,Variables=>{f2}) -- extract coefficients
\end{verbatim}

We see that there exist three formulae for each focal length, for example, 
{\small
%\begin{equation}
\begin{multline}
\label{eq:3}
f^2_2 = - \frac{c23_{0,1}}{c23_{0,0}} =  - \frac{\M{F}_{3,3} (\M{F}_{1,1} \M{F}_{2,3} \M{F}_{3,1}+\M{F}_{1,2} \M{F}_{2,3} \M{F}_{3,2}-\M{F}_{1,3} \M{F}_{2,1} \M{F}_{3,1}-\M{F}_{1,3} \M{F}_{2,2} \M{F}_{3,2})}{\splitdfrac{(\M{F}_{1,1}^2 \M{F}_{1,3} \M{F}_{2,3}-\M{F}_{1,1} \M{F}_{1,3}^2 \M{F}_{2,1}+\M{F}_{1,1} \M{F}_{2,1} \M{F}_{2,3}^2+\M{F}_{1,2}^2 \M{F}_{1,3} \M{F}_{2,3}}{-\M{F}_{1,2} \M{F}_{1,3}^2 \M{F}_{2,2}+\M{F}_{1,2} \M{F}_{2,2} \M{F}_{2,3}^2-\M{F}_{1,3} \M{F}_{2,1}^2 \M{F}_{2,3}-\M{F}_{1,3} \M{F}_{2,2}^2 \M{F}_{2,3})}}.
\end{multline}}
%\end{equation}}
It can be checked that this formula is equivalent to the Bougnoux formula by expressing the latter directly in terms of Fundamental matrix elements. The two other formulae are: 
{\small
\begin{multline}
\label{eq:2}
f^2_2 = - \frac{c22_{0,1}}{c22_{0,0}} =- \dfrac{\M{F}_{3,3} (\M{F}_{1,1} \M{F}_{3,1} \M{F}_{3,3}+\M{F}_{1,2} \M{F}_{3,2} \M{F}_{3,3}-\M{F}_{1,3} \M{F}_{3,1}^2-\M{F}_{1,3} \M{F}_{3,2}^2)}{\splitdfrac{(\M{F}_{1,1}^2 \M{F}_{1,3} \M{F}_{3,3}-\M{F}_{1,1} \M{F}_{1,3}^2 \M{F}_{3,1}+\M{F}_{1,1} \M{F}_{2,1} \M{F}_{2,3} \M{F}_{3,3}+\M{F}_{1,2}^2 \M{F}_{1,3} \M{F}_{3,3}}{-\M{F}_{1,2} \M{F}_{1,3}^2 \M{F}_{3,2}+\M{F}_{1,2} \M{F}_{2,2} \M{F}_{2,3} \M{F}_{3,3}-\M{F}_{1,3} \M{F}_{2,1} \M{F}_{2,3} \M{F}_{3,1}-\M{F}_{1,3} \M{F}_{2,2} \M{F}_{2,3} \M{F}_{3,2})}}.
\end{multline}}
{\small
\begin{multline}
\label{eq:1}
f^2_2 = - \frac{c21_{0,1}}{c21_{0,0}}  =-\dfrac{\M{F}_{3,3} (\M{F}_{2,1} \M{F}_{3,1} \M{F}_{3,3}+\M{F}_{2,2} \M{F}_{3,2} \M{F}_{3,3}-\M{F}_{2,3} \M{F}_{3,1}^2-\M{F}_{2,3} \M{F}_{3,2}^2)}{\splitdfrac{(\M{F}_{1,1} \M{F}_{1,3} \M{F}_{2,1} \M{F}_{3,3}-\M{F}_{1,1} \M{F}_{1,3} \M{F}_{2,3} \M{F}_{3,1}+\M{F}_{2,1}^2 \M{F}_{2,3} \M{F}_{3,3}-\M{F}_{2,1} \M{F}_{2,3}^2 \M{F}_{3,1}}{+\M{F}_{1,2} \M{F}_{1,3} \M{F}_{2,2} \M{F}_{3,3}-\M{F}_{1,2} \M{F}_{1,3} \M{F}_{2,3} \M{F}_{3,2}+\M{F}_{2,2}^2 \M{F}_{2,3} \M{F}_{3,3}-\M{F}_{2,2} \M{F}_{2,3}^2 f_{3,2})}}.
\end{multline}}

The formulae for the other focal length $f_1$ may be obtained by transposing the fundamental matrix. The undertaken analysis of the formulae has shown that generically they differ little in terms of stability and quality of estimations. However, as the denominators of the formulae are different, each of them has its own degeneracies, where another one may succeed instead.
We proceed to give examples of such situations.

It can be shown by substitution that the Bougnoux formula vanishes whenever either first or second column of $\mathtt{F}$ is the zero vector. Of the formulae we found, however, the third formula (Eq.~\ref{eq:1}) does not necessarily vanish when the first column of $\mathtt{F}$ is zero, and the second formula (Eq.~\ref{eq:2}) does not necessarily vanish when the second column is zero. When the third column is zero, all the formulae either vanish or become $p(\M{F})\,f_2^2=0$. The focal lengths cannot be reconstructed in this case per lemma~\ref{interlemma}, as $\mathtt{F}_{3,3}=0$. 

\begin{exmp}
\label{ex:tilt}
The pair of cameras $\M{P}_1$, $\M{P}_2$ both have their optical axes almost perpendicular to the baseline (tilted by an angle of $\pi \slash 10$) and experience the degeneration where the plane defined by the baseline and the optical axis of one camera is perpendicular to the plane defined by the baseline and optical axis of the other camera~\cite{HartZiss}. The calibration matrices are identity matrices. More precisely, the fundamental matrix of the pair is:

\begin{align*}
    & R^{tilt} = \mat{ccc}{\cos (\pi \slash 10) & 0 & \sin (\pi \slash 10) \\ 0 & 1 & 0 \\ -\sin (\pi \slash 10) & 0 & \cos (\pi \slash 10)} \\
    & \M{F}= \M{R} * \cross{t} = R^{tilt}    
    \mat{ccc}{1 & 0 & 0 \\ 0 & 0 & -1 \\ 0 & 1 & 0}
    R^{tilt}
    \mat{ccc}{0 & 0 & 0 \\ 0 & 0 & -1 \\ 0 & 1 & 0} 
    \approx \mat{ccc}{0 & 0.3 & 0.3 \\ 0 & 0.9 & 0 \\ 0 & 0.1 & 0.9 }
\end{align*}


By substitution, the first and second (Eqs.~\ref{eq:2},~\ref{eq:3}) vanish on this example,, i.e., assume the form $  0 \; f_2^2 = 0 $, but the third formula (Eq.~\ref{eq:1}) successfully gives a correct focal length estimate.

\end{exmp}

The example~\ref{ex:tilt} shows that there among all configurations known to be degenerate for the Bougnoux formula, there exist some that can be successfully solved by using the correct formula of the existing three. We see that specifically the degeneration  when the plane defined by the baseline and the optical axis of one camera is perpendicular to the plane defined by the baseline and optical axis of the other camera can be, at least in some cases, solved. We also see that in the example~\ref{ex:tilt} the configuration corresponds to a fundamental matrix which has the first column equal to zero vector.
%We also conjecture that this happens when first or second column is zero occurs


Note that there still remains a degeneracy when each of the three denominators vanishes. When this happens, the focal lengths cannot be reconstructed from the matrix $\mathtt{F}$, as there are no constraints on them (they can have any value). The next example shows when this degeneracy might occur.

\begin{exmp}
\label{ex:fail}
The pair of cameras $\M{P}_1$, $\M{P}_2$ are similar as in the example~\ref{ex:tilt}, but without tilt, i.e., they both have their optical axes perpendicular to the baseline. They also experience the degeneration where the plane defined by the baseline and the optical axis of one camera is perpendicular to the plane defined by the baseline and optical axis of the other camera. The calibration matrices are identity matrices. More precisely, the fundamental matrix of the pair is:

\begin{equation*}
    \M{F}= \M{R} * \cross{t} = \mat{ccc}{1 & 0 & 0 \\ 0 & 0 & -1 \\ 0 & 1 & 0}  \mat{ccc}{0 & 0 & 0 \\ 0 & 0 & -1 \\ 0 & 1 & 0} = \mat{ccc}{0 & 0 & 0 \\ 0 & -1 & 0 \\ 0 & 0 & -1}
\end{equation*}

By substitution, all three formulae (Eqs.~\ref{eq:1}~\ref{eq:2},~\ref{eq:3}) vanish on this example, i.e., assume the form $  0 \; f_2^2 = 0 $

\end{exmp}


Based on the examples~\ref{ex:tilt},~\ref{ex:fail} we \textit{conjecture} that the remaining degeneracy is the case when the both camera have their optical axes perpendicular to the baseline, and also experience the degeneration where the plane defined by the baseline and the optical axis of one camera is perpendicular to the plane defined by the baseline and optical axis of the other camera. In this case all three formulae should fail.

\section{The ratio formula}
We present a formula to compute directly the ratio $r = f_2 \slash f_1$. Note that unlike focal length formulae, there can be only one algebraically independent formula for computing the ratio.

% tex.stackexchange.com/questions/180917/breaking-the-equation-inside-a-fraction

{\tiny
\begin{equation}
r = \frac{f_2}{f_1} =  
\dfrac{\splitdfrac{\splitdfrac{\M{F}_{1,1}^2 \M{F}_{3,1}^2 \M{F}_{3,3}+2 \M{F}_{1,1} \M{F}_{1,2} \M{F}_{3,1} \M{F}_{3,2} \M{F}_{3,3}-\M{F}_{1,1} \M{F}_{1,3} \M{F}_{3,1}^3-\M{F}_{1,1} \M{F}_{1,3} \M{F}_{3,1} \M{F}_{3,2}^2}{+\M{F}_{1,2}^2 \M{F}_{3,2}^2 \M{F}_{3,3}-\M{F}_{1,2} \M{F}_{1,3} \M{F}_{3,1}^2 \M{F}_{3,2}-\M{F}_{1,2} \M{F}_{1,3} \M{F}_{3,2}^3+\M{F}_{2,1}^2 \M{F}_{3,1}^2 \M{F}_{3,3}+2 \M{F}_{2,1} \M{F}_{2,2} \M{F}_{3,1} \M{F}_{3,2} \M{F}_{3,3}}}{-\M{F}_{2,1} \M{F}_{2,3} \M{F}_{3,1}^3-\M{F}_{2,1} \M{F}_{2,3} \M{F}_{3,1} \M{F}_{3,2}^2+\M{F}_{2,2}^2 \M{F}_{3,2}^2 \M{F}_{3,3}-\M{F}_{2,2} \M{F}_{2,3} \M{F}_{3,1}^2 \M{F}_{3,2}-\M{F}_{2,2} \M{F}_{2,3} \M{F}_{3,2}^3}}{\splitdfrac{\splitdfrac{\M{F}_{1,1}^2 \M{F}_{1,3}^2 \M{F}_{3,3}-\M{F}_{1,1} \M{F}_{1,3}^3 \M{F}_{3,1}+2 \M{F}_{1,1} \M{F}_{1,3} \M{F}_{2,1} \M{F}_{2,3} \M{F}_{3,3}-\M{F}_{1,1} \M{F}_{1,3} \M{F}_{2,3}^2 \M{F}_{3,1}}{+\M{F}_{1,2}^2 \M{F}_{1,3}^2 \M{F}_{3,3}-\M{F}_{1,2} \M{F}_{1,3}^3 \M{F}_{3,2}+2 \M{F}_{1,2} \M{F}_{1,3} \M{F}_{2,2} \M{F}_{2,3} \M{F}_{3,3}-\M{F}_{1,2} \M{F}_{1,3} \M{F}_{2,3}^2 \M{F}_{3,2}}}{-\M{F}_{1,3}^2 \M{F}_{2,1} \M{F}_{2,3} \M{F}_{3,1}-\M{F}_{1,3}^2 \M{F}_{2,2} \M{F}_{2,3} \M{F}_{3,2}+\M{F}_{2,1}^2 \M{F}_{2,3}^2 \M{F}_{3,3}-\M{F}_{2,1} \M{F}_{2,3}^3 \M{F}_{3,1}+\M{F}_{2,2}^2 \M{F}_{2,3}^2 \M{F}_{3,3}-\M{F}_{2,2} \M{F}_{2,3}^3 \M{F}_{3,2}}}
\end{equation}
}

Computing the ratio directly from $\mathtt{F}$ can offer  greater speed and more stability than computing it from the focal lengths. Indeed, we observed that for close to degenerate situations, the formula,  although marginally, is a better estimator. 

\section{Computing Fundamental matrices}

We show in detail how the 7pt solver works from the viewpoint of algebraic geometry.

% In the next snippet of code we simulate a random scene\footnote{The simulated points may occur in a singular configuration. If this happens, rank of the matrix $\mathtt{A}$  will not be 7, and there will be an infinite number of possible matrices $\mathtt{F}$. This, however, can almost never happen.}

% \begin{verbatim}
% fl1=1500
% fl2=2000
% K1 = matrix{{fl1,0,0},{0,fl1,0},{0,0,1}}
% K2 = matrix{{fl2,0,0},{0,fl2,0},{0,0,1}}
% R2 = matrix{{0,1,0},{-3/5,0,4/5},{4/5,0,3/5}} * matrix{{1,0,0},{0,3/5,-4/5},{0,4/5,3/5}} ** 
% R1 = id_(R^3)
% points=7
% X  = matrix(fillMatrix(mutableMatrix(QQ,4,points)))
% X =  mutableMatrix for i from 0 to points-1 list (1/X_i_3)*(X_i)
% for i from 0 to points-1 do X_(2,i)=X_(2,i)+3
% X = matrix X
% t1 = transpose matrix{{0,0,0}}
% t2 = transpose matrix{{-1,-1,-1}}
% P1=K1*(R1| -R1*t1)
% P2=K2*(R2| -R2*t2)

% u1=P1*X
% u2=P2*X

% B = transpose matrix for i from 0 to points-1 list u1_i ** u2_i 

% \end{verbatim}


We first simulate a random set of correspondences. We skip this code for brevity and assume that the matrix $\mathtt{B}$ is a matrix from the 7pt algorithm (algorithm \ref{7pt}), i.e., the matrix, the right null space of which is the linear space of all (vectorized) matrices satisfying 7 certain epipolar constraints~\ref{eq:epipolar}. 

In the next snippet we show detailed analysis of the computation results. The variety $\textbf{V}(GBs)$ is the variety of all possible focal lengths that explain the simulated correspondences.

\begin{verbatim}
m = transpose matrix {{f11,f12,f13,f21,f22,f23,f31,f32,f33}}
rank B
eB = B*m
IB = minors(1,eB)
GBs = Gs + IB
gGBS=mingens gb GBs
dim GBs, codim GBs, degree GBs
pGBS = minimalPrimes GBs
\end{verbatim}

Ideal $IB$ contains the epipolar constraints. By adding together the ideals $Gs$ and $IB$ we combine the constraints and obtain an ideal that corresponds to a variety of solutions of the 7pt solver for our correspondences.

Variety $\textbf{V}(GBs)$ can consist of either 5 or 13 components, one of dimension 2, and the rest of dimension 0. We will show the significance of this fact and what are these components corresponding to.

We provide an example of what the minimal primes (ideals, corresponding to component varieties) may look like. 

\begin{exmp}
The \textit{minimal primes} of an ideal containing seven epipolar constraints and the rank constraint. The actual minimal primes are 6 ideals, which correspond to 5 component varieties over $\Q$. One of the prime ideals does not correspond to any variety over $\Q$, but would decompose into 8 point varieties over $\C$. We exclude this ideal from the example for the sake of brevity. The remaining 5 ideals are as follows:
\begin{enumerate}
    \item   $\langle\M{F}_{3,3}$, $\M{F}_{3,2}$, $\M{F}_{3,1}$, $\M{F}_{2,3}$, $\M{F}_{2,2}$, $\M{F}_{2,1}$, $\M{F}_{1,3}$, $\M{F}_{1,2}$, $\M{F}_{1,1}\rangle$
    \item $\langle16000 \M{F}_{3,2} + 31 \M{F}_{3,3}$, $3200 \M{F}_{3,1} + 3 \M{F}_{3,3}$, $12000 \M{F}_{2,3} + 11 \M{F}_{3,3}$, $8000000 \M{F}_{2,2} - 9 \M{F}_{3,3}$, $1200000 \M{F}_{2,1} + \M{F}_{3,3}$, $4000 \M{F}_{1,3} - \M{F}_{3,3}$, $       6000000 \M{F}_{1,2} - \M{F}_{3,3}$, $4800000 \M{F}_{1,1} - 7 \M{F}_{3,3}$, $f_2 + 1500$, $f_1 - 2000\rangle$ 
    \item  $\langle16000 \M{F}_{3,2} + 31 \M{F}_{3,3}$, $3200 \M{F}_{3,1} + 3 \M{F}_{3,3}$, $12000 \M{F}_{2,3} + 11 \M{F}_{3,3}$, $8000000 \M{F}_{2,2} - 9 \M{F}_{3,3}$, $1200000 \M{F}_{2,1} + \M{F}_{3,3}$, $4000 \M{F}_{1,3} - \M{F}_{3,3}$, $6000000 \M{F}_{1,2} - \M{F}_{3,3}$, $4800000 \M{F}_{1,1} - 7 \M{F}_{3,3}$, $f_2 + 1500$, $f_1 +        2000\rangle$
    \item $\langle16000 \M{F}_{3,2} + 31 \M{F}_{3,3}$, $3200 \M{F}_{3,1} + 3 \M{F}_{3,3}$, $12000 \M{F}_{2,3} + 11 \M{F}_{3,3}$, $8000000 \M{F}_{2,2} - 9 \M{F}_{3,3}$, $1200000 \M{F}_{2,1} + \M{F}_{3,3}$, $4000 \M{F}_{1,3} - \M{F}_{3,3}$, $6000000 \M{F}_{1,2} - \M{F}_{3,3}$, $4800000 \M{F}_{1,1} - 7 \M{F}_{3,3}$, $f_2 -        1500$, $f_1 + 2000\rangle$
    \item $\langle16000 \M{F}_{3,2} + 31 \M{F}_{3,3}$, $3200 \M{F}_{3,1} + 3 \M{F}_{3,3}$, $12000 \M{F}_{2,3} + 11 \M{F}_{3,3}$, $8000000 \M{F}_{2,2} - 9 \M{F}_{3,3}$, $1200000 \M{F}_{2,1} + \M{F}_{3,3}$, $4000 \M{F}_{1,3} - \M{F}_{3,3}$, $6000000 \M{F}_{1,2} - \M{F}_{3,3}$, $4800000 \M{F}_{1,1} - 7        \M{F}_{3,3}$, $f_2 - 1500$, $f_1 - 2000\rangle$
\end{enumerate}
\end{exmp}


The first ideal corresponds to the component of dimension 2. It represents the degenerate situation when  the matrix $\M{F}$ is the zero matrix. The ideal also does not contain any polynomial in $f_1$ or $f_2$, so neither of focal lengths can be determined as there are no constraints on them. Because of this, the component has dimension 2, its two degrees of freedom are $f_1$ and $f_2$.

% I might be wrong writing this one
% The first component contains the polynomials $f_1$ and $f_2$, which means that  it corresponds to a degenerate case when $f_1=0, f_2=0$. Technically, such a fundamental matrix satisfies all constraints and is an example of a matrix having rank less than two (see the discussion in section \ref{seq:failure}). This case could be avoided by using  unknowns $w_1=\frac{1}{f_1}, w_2=\frac{1}{f_2}$. \TP{I tried it and there is a case f(F,w1,w2)=0. Which still needs to be analyzed.} \OR{what is f?}\TP{f stands here for a polynomial, its really a bit confusing}

The 12 (or 4) other components are of dimension 0 are point varieties (all  unknowns are determined there). They  can be divided into 3 groups (or 1 group), which correspond(s) to 3 different Fundamental matrices (a single matrix). Each group includes 4 point varieties in form $f_1 = \pm  C_1, f_2 \pm C_2$ where $C_1, C_2$ are constants. The reason for this is that the focal length actually only enters the Bougnoux formula, as well as formulae we derived (Eqs.~\ref{eq:1},~\ref{eq:2},~\ref{eq:3}) in the second degree, so it is impossible to determine it's sign\footnote{Of course for any practical application the situation is unambiguous, as the positive sign should always be chosen.}.

We see that each group corresponds to one of the three fundamental matrices as returned by 7pt algorithm. Sometimes, however, two of them would be complex. In this case we will get only one group, as our code considers varieties over real numbers. This explains why sometimes we can get only 4 components with nonzero $f_1, f_2$ instead of 12.


\section{Conclusions}

We have shown how Bougnoux formula and 7pt algorithm work in terms of algebraic geometry. We have derived 2 new formulae for computing focal lengths from fundamental matrix, and a formula for computing ratio $r = f_2 \slash f_1$. No more algebraically independent formulae can be derived for this problem.
We have shown that a known degeneracy \cite{HartZiss} can be partially avoided by using the right focal length formula of the three. The degeneracy reduces to the case when all three formulae fail.

We  confirmed that three and only three fundamental matrices can explain 7 correspondences. 
We have shown that the trace constraint is redundant while solving camera relative pose with two unknown focal lengths.